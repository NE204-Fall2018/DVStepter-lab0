The purpose of Lab0 is to practice generating lab reports using the reproducible work flow required for NE204: Advanced Concepts in Radiation Detection. Energy calibration of a High Purity Germanium (HPGe) detector, a ubiquitous task in radiation detection laboratories, was used for practicing the work flow. Energy calibration is a necessary step for every radiation detector. This is especially true for HPGe Detectors that are able to achieve high-energy resolution beyond what other detectors, such as scintillators, are capable of. Precise calibration ensures the accuracy and the quality of the results. Without a proper calibrations, there is no way to tell how the channels/bins from a Multi-Channel Analyzer (MCA) correlate to energy. In Lab0 this energy calibration was done using a two-point linear calibration between two gamma-ray photopeaks of $^{137}$Cs and $^{241}$Am. Subsequently, the calibration model was "verified" by applying it to the gamma-ray spectrum from $^{133}$Ba, which has five distinct photopeaks. This report details the process and results of a two-point linear calibration.
