The data for this lab was collected by collected by Dr. Ross Barnowski. Dr. Barnowski used a HPGe detector and a 13-bit resolution MCA, yielding 8192-bin spectra. It is assumed that each of these measurements were taken with each source at the same location and distance from the detector. The sources used to generate the spectrum are shown in Table \ref{tab:sources}.

\begin{table}[H]
\centering
\caption{Sources for Lab0}
\label{tab:sources}
\begin{tabular}{@{}lllll@{}}
\toprule
$^{241}$Am & $^{133}$Ba & $^{60}$Co & $^{137}$Cs & $^{152}$Eu \\ \bottomrule
\end{tabular}
\end{table}

The energy calibration was performed using a two-point linear fit between $^{137}$Cs and $^{241}$Am. To perform the calibration, a simple python peak detection script searched the raw spectrum data of $^{137}$Cs and $^{241}$Am looking for the channel containing largest number of counts (photopeak) within the spectrum. The program iterated over the spectrum by channel trying to identify a large delta (40000) in the counts. Because $^{137}$Cs and $^{241}$Am have single gamma ray photopeaks, a large delta ensures no other noise in the signal could be misconstrued as photopeaks.

Once the $^{137}$Cs and $^{241}$Am peaks were discovered, the python function polyfit was used to plot a linear line. The inputs to polyfit were the channel of the peak and the actual gamma-ray photopeak energy as listed in the LBNL Nuclear Data Search~\cite{chu_ekstrom_firestone_1999}. The polyfit output was a slope and intercept that can be used to translate channel to energy. Next, the peak detection script was used to find the multiple photopeaks in the $^{133}$Ba sprectrum. Lastly, the polyfit "calibration" was applied to the $^{133}$Ba photopeaks and spectrum to calculate approximation error and display the full $^{133}$Ba spectrum.
